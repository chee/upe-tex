\chapter{The File System}

Everything in the UNIX system is a file. That is less of an oversimplification
than you might think. When the first version of the system was being designed,
before it even had a name, the discussions focused on the structure of a file
system that would be clean and easy to use. The file system is central to the
success and convenience of the UNIX system. It is one of the best examples of
the ``keep it simple'' philosophy, showing the power achieved by careful
implementation of a few well-chosen ideas.

To talk comfortably about commands and their interrelationships, we need a good
background in the structure and outer workings of the file system. This chapter
covers most of the details of using the file system --- what files are, how they
are represented, directories and the file system hierarchy, permissions, inodes
(the system's internal record of files) and device files. Because most use of
the UNIX system deals with manipulating files, there are many commands for file
manipulation and rearrangement; this chapter introduces the more commonly used
ones.

\section{The basics of files}
\section{What's in a file?}
\section{Directories and filenames}
\section{Permissions}
\section{Inodes}
\section{The directory hierarchy}
\section{Devices}
