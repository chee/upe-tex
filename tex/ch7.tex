\chapter{UNIX System Calls}
This chapter concentrates on the lowest level of interaction with the UNIX
operating system --- the system calls. These are the entries to the kernel. They
are the facilities that the operating system provides; everything else is built
on top of them.

We will cover several major areas. First is the I/O system, the foundation
beneath library routines like fopen and putc. We'll talk more about the file
system as well, particularly directories and inodes. Next comes a discussion of
processes --- how to run programs from within a program. After that we will talk
about signals and interrupts: what happens when you push the DELETE key, and how
to handle that sensibly in a program.

As in Chapter 6, many of our examples are useful programs that were not part of
the 7th Edition. Even if they are not directly helpful to you, you should learn
something from reading them, and they might suggest similar tools that you could
build for your system.

Full details on the system calls in Section 2 of the UNIX Programmer's Manual;
this chapter describes that most important parts, but makes no pretense of
completeness.
\section{Low-level I/O}
\section{File system: directories}
\section{File system: inodes}
\section{Processes}
\section{Signals and interrupts}
