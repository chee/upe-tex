\chapter{UNIX for Beginners}


What is ``UNIX''? In the narrowest sense, it is a time-sharing operating system
kernel: a program that controls the resources of a computer and allocates them
among its users. It lets users run their programs; it controls the peripheral
devices (discs, terminals, printers, and the like) connected to the machine; and
it provides a file system that manages the long-term storage of information such
as programs, data, and documents.

In a broader sense, ``UNIX'' is often taken to include not only the kernel, but
also essential programs like compilers, editors, command languages, programs for
copying and printing files, and so on.

Still more broadly, ``UNIX'' may even include programs developed by you or other
users to be run on your system, such as tools for document preparation, routines
for statistical analysis, and graphics packages.

Which of these uses of the name ``UNIX'' is correct depends on which level of
the system you are considering. When we use ``UNIX'' in the rest of this book,
context should indicate which meaning is implied.

The UNIX system sometimes looks more difficult than it is --- it's hard for a
newcomer to know how to make the best use of the facilities available. But
fortunately it's not hard to get started --- knowledge of only a few programs
should get you off the ground. This chapter is meant to help you to start using
the system as quickly as possible. It's an overview, not a manual; we'll cover
most of the material again in more detail in later chapters. We'll talk about
these major areas:

\begin{itemize}
\item basics --- logging in and out, simple commands, correcting typing
  mistakes, mail, inter-terminal communication.
\item day-to-day use --- files and file system, printing files, directories,
  commonly-used commands.
\item the command interpreter or shell --- filename shorthands, redirecting
  intput and output, pipes, setting erase and kill characters, and defining your
  own search path for commands.
\end{itemize}

If you've used a UNIX system before, most of this chapter should be familiar;
you might want to skip straight to Chapter 2.

You will need a copy of the \textit{UNIX Programmer's Manual}, even as you read
this chapter; it's often easier for us to tell you to read about something in
the manual than to repeat its contents here. This book is not supposed to
replace it, but to show you how to make best use of the commands described in
it. Furthermore, there may be differences between what we say here and what is
true on your system. The manual has a permuted index at the beginning that's
indispensable for finding the right programs to apply to a problem; learn to use
it.

Finally, a word of advice: don't be afraid to experiment. If you are a beginner,
there are very few accidental things you can do to hurt yourself or other
users. So learn how things work by trying them. This is a long chapter, and the
best way to read it is a few pages at a time, trying things out as you go.


\section{Getting Started}
\subsection{Some prerequisites about terminals and typing}

To avoid explaining everything about using computers, we must assume you have
some familiarity with computer terminals and how to use them. If any of the
following statements are mystifying, you should ask a local expert for help.

The UNIX system is \textit{full duplex}: the characters you type on the keyboard
are sent to the system, which sends them back to the terminal to be printed on
the screen. Normally, this \textit{echo} process copies the characters directly
to the screen, so you can see what you are typing, but sometimes, such as when
you are typing a secret password, the echo is turned off so the characters do
not appear on the screen.

Most of the keyboard characters are ordinary printing characters with no special
significance, but a few tell the computer how to interpret your typing. By far
the most important of these is the RETURN key. The RETURN key signifies the end
of a line of input; the system echoes it by moving the terminal's cursor to the
beginning of the next line on the screen. RETURN must be pressed before the
system will interpret the characters you have types.

RETURN is an example of a \textit{control character} --- an invisible character
that controls some aspect of input and output on the terminal. On any reasonable
terminal, RETURN has a key of its own, but most control characters do
not. Instead, they must be typed by holding down the CONTROL key, sometimes
called CTL or CNTL or CTRL, then pressing another key, usually a letter. For
example, RETURN may be typed by pressing the RETURN key or, equivalently,
holding down the CONTROL key and typing an `m'. RETURN might therefore be called
a control-m, which we will write as \textit{ctl}-m. Other control characters
include \textit{ctl}-d, which tells a program that there is no more input;
\textit{ctl}-g, which rings the bell on the terminal; \textit{ctl}-h, often
called backspace, which can be used to correct typing mistakes; and
\textit{ctl}-i, often called tab, which advances the cursor to the next tab
stop, much as on a regular typewriter. Tab stops on UNIX systems are eight
spaces apart. Both the backspaces and tab characters have their keys on the
terminals.

Two other keys have special meaning: DELETE, sometimes called RUBOUT or some
abbreviation, and BREAK, sometimes called INTERRUPT. On most UNIX systems, the
DELETE key stops a program immediately, without waiting for it to finish. On
some systems, \textit{ctl}-c provides this device. And on some systems,
depending on how the terminals are connected, BREAK is synonym for DELETE or
\textit{ctl}-c.


\subsection{A Session with UNIX}

Let's begin with an annotated dialog between you and your UNIX system.
Throughout the examples in this book, what you type is printed in slanted
letters, computer responses are in typewriter-style characters, and explanations
are in \textit{italics}.

\begin{verbatim}
        Establish a connection: dial a phone or turn on a switch as necessary.
        Your system should say
        login: you                 Type your name, then press RETURN
        Password:                  Your password won't be echoed as you type it
        You have mail.             There's mail to be read after you log in
        $                          The system is now ready for your commands
        $                          Press RETURN a couple of times
        $ date                     What is the date and time?
        Sun Sep 25 23:02:57 EDT 1983
        $ who                      Who's using the computer?
        jlb      ttyO    Sep 25 13:59
        you      tty2    Sep 25 23:01
        mary     tty4    Sep 25 19:03
        doug     tty5    Sep 25 19:22
        egb      tty7    Sep 25 17:17
        bob      tty8    Sep 25 20:48
        $ mail                     Read your mail
        From doug Sun Sep 25 20:53 EDT 1983
        give me a call sometime monday

        ?                          RETURN moves on to the next message
        From mary Sun Sep 25 19:07 EDT 1983     Next message

        ? d                        Delete this message
        $                          No more mail
        $ mail mary                Send mail to mary
        lunch at 12 is fine
        ctl-d                      End of mail
        $                          Hang up phone or turn off terminal
                                   and that's the end
\end{verbatim}
Sometimes that's all there is to a session, though occasionally people do some
work too. The rest of this section will discuss the session above, plus other
programs that make it possible to do useful things.


\subsection{Logging in}

You must have a login name and password, which you can get from your system
administrator. The UNIX system is capable of dealing wit a wide variety of
terminals, but it is strongly oriented towards devices with lower case; case
distinctions matter! If your terminal produces only upper case (like some video
and portable terminals), life will be so difficult that you should look for
another terminal.

Be sure the switches are set appropriately on your device: upper and lower case,
full duplex, and any other settings that local experts advise, such as the
speed, or \textit{baud rate}. Establish a connection using whatever magic is
needed for your terminal; this may involve dialing a telephone or merely
flipping a switch. In either case, the system should type
\begin{verbatim}
        login:
\end{verbatim}
If it types garbage, you may be at the wrong speed; check the speed setting and
other switches. If that fails, press the BREAK or INTERRUPT key a few times,
slowly. If nothing produces a login message, you will have to get help.

When you get the \verb=login:= message, type your login name \textit{in lower
  case}. Follow it by pressing RETURN. If a password is required, you will be
asked for it, and printing will be turned off while you type it.

The culmination of your login efforts is a \textit{prompt}, usually a single
character, indicating that the system is ready to accept commands from you. The
prompt is mostly likely to be a dollar sign \verb=$= or a percent sign \verb=%=,
but you can change it to anything you like; we'll show you how a little
later. The prompt is actually printed by a program called the \textit{command
  interpreter} or \textit{shell}, which is your main interface to the system.

There may be a message of the day just before the prompt, or a notification that
you have mail. You may also be asked what kind of terminal you are using; your
answer helps the system to use any special properties the terminal might have.


\subsection{Typing commands}

Once you receive the prompt, you can type commands, which are requests that the
system do something. We will use \textit{program} as a synonym for
\textit{command}. When you see the prompt (let's assume it's \verb=$=), type
date and press RETURN. the system should replay with the date and time, then
print another prompt, so the whole transaction will look like this on your
terminal:
\begin{verbatim}
        $ date
        Mon Sep 26 12:20:57 EDT 1983
        $
\end{verbatim}
Don't forget RETURN, and don't type the \verb=$=. If you think you're being
ignored, press RETURN; something should happen. RETURN won't be mentioned again,
but you need it at the end of every line.

The next command to try is who, which tells you everyone who is currently logged
in:
\begin{verbatim}
        $ who
        rim     tty0     Sep 26 11:37
        pjw     tty4     Sep 26 11:30
        gerard  tty7     Sep 26 10:27
        mark    tty9     Sep 26 07:59
        you     ttya     Sep 26 12:20
        $
\end{verbatim}
The first column is the user name. The second is the system's name for the
connection being used (``tty'' stands for ``teletype,'' an archaic synonym for
``terminal''). The rest tells when the user logged on. You might also try
\begin{verbatim}
        $ who am i
        you     ttya     Sep 26 12:20
        $
\end{verbatim}

If you make a mistake typing the name of a command, and refer to a non-existent
command, you will be told that no command of that name can be found:
\begin{verbatim}
        $ whom                     Misspelled command name ...
        whom: not found            ... so system didn't know how to run it
        $
\end{verbatim}
Of course, if you inadvertently type the name of an actual command, it will run,
perhaps with mysterious results.


\subsection{Strange terminal behavior}

Sometimes your terminal will act strangely, for example, each letter may be
typed twice, or RETURN may not put the cursor at the first column of the next
line. You can usually fix this by turning the terminal off and on, or by logging
out and logging back in. Or you can read the description of the command
\texttt{stty} (``set terminal options'') in Section 1 of the manual. To get
intelligent treatment of tab characters if your terminal doesn't have tabs, type
the command
\begin{verbatim}
        $ stty -tabs
\end{verbatim}
and the system will convert tabs into the right number of spaces. If your
terminal does have computer-settable tab stops, the command tabs will set them
correctly for you. (You may actually have to say
\begin{verbatim}
        $ tabs terminal-type
\end{verbatim}
to make it work --- see the \texttt{tabs} command description in the manual.)


\subsection{Mistakes in typing}
If you make a typing mistake, and see it before you have pressed RETURN, there
are two ways to recover, \textit{erase} characters one at a time or
\textit{kill} the whole line and re-type it.

If you type the line \textit{kill} character, by default an at-sign \verb=@=, it
causes the whole line to be discarded, just as if you'd never typed it, and
starts you over on a new line:
\begin{verbatim}
        $ ddtae@                    Completely botched; start over
        date                          on a new line
        Mon Sep 26 12:23:39 EDT 1983
        $
\end{verbatim}
The sharp character \verb=#= erases the last character typed; each \verb=#=
erases one more character, back to the beginning of the line (but not
beyond). So if you type badly, you can correct as you go:
\begin{verbatim}
        $ dd#atte##e                Fix it as you go
        Mon Sep 26 12:23:39 EDT 1983
        $
\end{verbatim}
The particular erase and line kill characters are very system dependent. On many
systems (including the one we use), the erase character has been changed to
backspace, which works nicely on video terminals. You can quickly check which is
the case on your system:
\begin{verbatim}
        $ datee<-                              Try <-
        datee<-: not found                     It's not <-
        $ datee#                               Try #
        Mon Sep 26 12:23:39 EDT 1983           It is #
        $
\end{verbatim}
(We printed the backspace character \verb=<-= as you can see it.) Another common
choice is \textit{ctl}-u for line kill.

We will use the sharp as the erase character for the rest of this section
because it's visible, but make the mental adjustment if your system is
different. Later on, in ``tailoring the environment,'' we will tell you how to
set the erase and line kill characters to whatever you like, once and for all.

What if you must enter an erase or line kill character as part of the text? If
you precede either \verb=#= or \verb=@= by a backslash \verb=\=, it loses its
special meaning. So to enter a \verb=#= or \verb=@=, type \verb=\#= or
\verb=\@=. The system may advance the terminal's cursor to the next line after
your \verb=@=, even if it was preceded by a backslash. Don't worry --- the
at-sign has been recorded.

The backslash, sometimes called the \textit{escape character}, is used
extensively to indicate that the following character is in some way special. To
erase a backslash, you have to type two erase characters \verb=\##=. Do you see
why?

The characters you type are examined and interpreted by a sequence of programs
before they reach their destination, and exactly how they interpreted depends
not only on where they end up but how got there.

Every character you type is immediately echoed to the terminal, unless echoing
is turned off, which is rare. Until you press RETURN, the characters are held
temporarily by the kernel, so typing mistakes can be corrected with the erase
and line kill characters. When an erase or line kill character is preceded by a
backslash, the kernel discards the backslash and holds the following character
without interpretation.

When you press RETURN, the characters being held are sent to the program that is
reading from the terminal. That program may in turn interpret the characters in
special ways; for example, the shell turns off any special interpretation of a
character if it is preceded by a backslash. We'll come back to this in Chapter
3. For now, you should remember that the kernel processes erase and line kill,
and backslash only if it precedes erase or line kill; whatever characters are
left after that may be interpreted by other programs as well.

Exercise 1-1. Explain what happens with
\begin{verbatim}
        $ date\@
\end{verbatim}

Exercise 1-2. Most shells (although not the 7th Edition shell) interpret
\verb=#= as introducing a comment, and ignore all text from the \verb=#= to the
end of the line. Given this, explain the following transcript, assuming your
erase character is also \verb=#=
\begin{verbatim}
        $ date
        Mon Sep 26 12:39:56 EDT 1983
        $ #date
        Mon Sep 26 12:40:21 EDT 1983
        $ \#date
        $ \\#date
        #date : not found
        $
\end{verbatim}


\subsection{Type-ahead}
The kernel reads what you type as you type it, even if it's busy with something
else, so you can type as fast as you want, whenever you want, even when some
command is printing at you. If you type while the system is printing, your input
characters will appear intermixed with the output characters, but they will be
stored away and interpreted in the correct order. You can type commands one
after another without waiting for them to finish or even to begin.


\subsection{Stopping a program}
You can stop most commands by typing the character DELETE. The BREAK key found
on most terminals may also work, although this is system dependent. In a few
programs, like text editors, DELETE stops whatever the program is doing but
leaves you in that program. Turning off the terminal or hanging up the phone
will stop most programs.

If you just want output to pause, for example to keep something critical from
disappearing off the screen, type \textit{ctl}-s. The output will stop almost
immediately; you program is suspended until you start it again. When you want to
resume, type \textit{ctl}-q.


\subsection{Logging out}
The proper way to log out is to type \textit{ctl}-d instead of a command; this
tells the shell that there is no more input. (How this actually works will be
explained in the next chapter.) You can usually just turn off the terminal or
hang up the phone, but whether this really logs you out depends on your system.


\subsection{Mail}
The system provides a postal system for communicating with other users, so some
day when you log in, you will see the message
\begin{verbatim}
        You have mail
\end{verbatim}
before the first prompt. To read your mail, type
\begin{verbatim}
        $ mail
\end{verbatim}
your mail will be printed, one message at a time, most recent first. After each
item, \texttt{mail} waits for you to say what to do with it. The two basic
responses are \texttt{d}, which deletes the message, and RETURN, which does not
(so it will still be there the next time you read your mail). Other responses
include \texttt{p} to reprint a message, \texttt{s} filename to save it in the
file you named, and \texttt{q} to quit from mail. (If you don't know what a file
is, think of it as a place where you can store information under a name of your
choice, and retrieve it later. Files are the topics of Section 1.2 and indeed of
much of this book.)

\texttt{mail} is one of those programs that is likely to differ from what we
describe here; there are some variants. Look in your manual for details.

Sending mail to someone is straightforward. Suppose it is to go to the person
with the login name \texttt{nico}. The easiest way is this:
\begin{verbatim}
        $ mail nico
        how type in the text of the letter
        on as many lines as you like ...
        After the last line of the letter
        type a control-d
        ctl-d
        $
\end{verbatim}
The \textit{ctl}-d signals the end of the letter by telling the \texttt{mail}
command that there is no more input. If you change your mind half-way through
composing the letter, press DELETE instead of \textit{ctl}-d. The half-formed
letter will be stored in a file called \texttt{dead.letter} instead of being
sent.


\subsection{Writing to other users}
If your UNIX system has multiple users, someday, out of the blue, your terminal will print something like

\begin{verbatim}
Message from mary tty7...
\end{verbatim}

accompanied by a startling beep.
Mary wants to write to you, but unless you take explicit action you won't be able to write back.
To respond, type

\begin{verbatim}
$ write mary
\end{verbatim}

This establishes a two-way communication path.
Now the lines that Mary types on her terminal will appear on yours and vice versa, although the path is slow, rather like talking to the moon.

If you are in the middle of something, you have to get to a state where you can type a command.
Normally, whatever program you are running has to stop or be stopped, but some programs, such as the editor and `write` itself, have a `!` command to escape temporarily to the shell - see Table 2 in Appendix 1.

The `write` command imposes no rules, so a protocol is needed to keep what you type from getting garbled up with what Mary types.
One convention is to take turns, ending each turn with `(o)`, which stands for "over," and to signal your intent to quit with `(oo)`, for "over and out."

\subsection{News}
\subsection{The manual}
\subsection{Computer-aided instruction}
\subsection{Games}

\section{Day-to-day use: files and common commands}
\section{More about files: directories}
\section{The shell}
\section{The rest of the UNIX system}
